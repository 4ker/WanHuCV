%%%%%%%%%%%%%%%%%%%%%%%%%%%%%%%%%%%%%%%%%
% "ModernCV" CV and Cover Letter
% LaTeX Template
% Version 1.1 (9/12/12)
%
% This template has been downloaded from:
% http://www.LaTeXTemplates.com
%
% Original author:
% Xavier Danaux (xdanaux@gmail.com)
%
% License:
% CC BY-NC-SA 3.0 (http://creativecommons.org/licenses/by-nc-sa/3.0/)
%
% Important note:
% This template requires the moderncv.cls and .sty files to be in the same
% directory as this .tex file. These files provide the resume style and themes
% used for structuring the document.
%
%%%%%%%%%%%%%%%%%%%%%%%%%%%%%%%%%%%%%%%%%

%----------------------------------------------------------------------------------------
%	PACKAGES AND OTHER DOCUMENT CONFIGURATIONS
%----------------------------------------------------------------------------------------

\documentclass[10pt,a4paper,sans]{moderncv} % Font sizes: 10, 11, or 12; paper sizes: a4paper, letterpaper, a5paper, legalpaper, executivepaper or landscape; font families: sans or roman
% moderncv version 1.5.1 (29 Apr 2013)


\moderncvstyle{banking} % CV theme - options include: 'casual' (default), 'classic', 'oldstyle' and 'banking'
\moderncvcolor{black} % CV color - options include: 'blue' (default), 'orange', 'green', 'red', 'purple', 'grey' and 'black'

\usepackage[adobefonts]{ctex} %中文支持
\setCJKmainfont{SimSun}

%\usepackage{lipsum} % Used for inserting dummy 'Lorem ipsum' text into the template

\usepackage[top=1cm,bottom=1cm]{geometry} % Reduce document margins
%\setlength{\hintscolumnwidth}{3cm} % Uncomment to change the width of the dates column
%\setlength{\makecvtitlenamewidth}{10cm} % For the 'classic' style, uncomment to adjust the width of the space allocated to your name

%----------------------------------------------------------------------------------------
%	NAME AND CONTACT INFORMATION SECTION
%----------------------------------------------------------------------------------------
\name{万}{虎}
% All information in this block is optional, comment out any lines you don't need
\title{个人简历}
\address{首都师范大学~信息工程学院}{海淀区, 北京市 100048}
\phone[mobile]{(+86)~152~10594789}
\email{mengyingchina@163.com}
%\homepage{www.wanhu.me}
%\social[twitter]{mengyingchina}
%\social[github]{mengyingchina}
%\extrainfo{additional information}
%\photo[70pt][0.4pt]{sunwisespace} % The first bracket is the picture height, the second is the thickness of the frame around the picture (0pt for no frame)
%\quote{"A witty and playful quotation" - John Smith}

%----------------------------------------------------------------------------------------

\begin{document}

\makecvtitle % Print the CV title

%----------------------------------------------------------------------------------------
%	POSITION APPLIED(CAREER OBJECTIVE)
%----------------------------------------------------------------------------------------
%\section{求职意向}
%%\subsection{求职意向}
%\cventry{期望月薪:  面议}{应聘职位:初级硬件工程师}{\includegraphics[scale=0.43]{sunwisespace.jpg}}{}{}{}

%----------------------------------------------------------------------------------------
%	EDUCATION SECTION
%----------------------------------------------------------------------------------------

\section{教育背景}

\cventry{2009--2013}{计算机科学与技术}{首都师范大学}{应届本科}{\textit{GPA -- 3.93/5}}{}{首都师范大学优秀毕业生}  % Arguments not required can be left empty

%----------------------------------------------------------------------------------------
%	WORK EXPERIENCE SECTION
%----------------------------------------------------------------------------------------

\section{项目经历}
%------------------------------------------------
\subsection{首都师范大学``本科生毕业论文(设计)"}

\cventry{2012--2013}{毕设小组负责人}{基于DDS 技术的AM/FM信号调制的设计及其FPGA 实现}{}{}{
\begin{itemize}
	\item 负责基于直接数字频率合成(DDS) 技术设计和实现幅度/频率调制功能
	\item 使用Altera 公司的Cyclone II 系列FPGA 芯片,完成系统编程
    \item 参与设计制作PCB 板,实现调制信号的采集和调频信号的发射
\end{itemize}
}
%------------------------------------------------
\subsection{首都师范大学``本科生科学研究与创业行动"项目}

\cventry{2012--2013}{项目负责人}{基于Nios II软核的FIR滤波器的设计}{校级}{}{
\begin{itemize}
	\item 负责解决FIR的IP核的设计,利用Audio ADC/DAC引脚来设计音频输入
	\item 负责解决FIR模块的例化,利用硬件描述语言实现整个音频滤波
    \item 基于Cyclone II EP2C70F896C6处理器实现FIR/IIR滤波器
\end{itemize}
}

%------------------------------------------------
%------------------------------------------------
\subsection{``北京市大学生科学研究与创业行动计划"项目}

\cventry{2011--2012}{项目成员}{功耗和温度感知的多核操作系统研究}{市级}{}{
\begin{itemize}
\item 参与设计实现一种Cache感知的调度算法(CAS),负责算法在Linux环境下的测试仿真数据处理
\item 项目获得\textbf{``北京市大学生科学研究与创业行动计划"二等奖}
\item 研究论文《面向多核处理器系统的Cache 感知调度算法》发表在中文核心期刊《小型微型计算机系统》
\end{itemize}
}
%------------------------------------------------
%------------------------------------------------
\subsection{首都师范大学``实验室开放基金"项目}

\cventry{2011--2012}{项目负责人}{网络工程创新实验设计——基于Hadoop的海量数据应用研究}{校级}{}{
\begin{itemize}
\item 带领小组对Hadoop大规模数据排序算法TeraSort进行分析并作为基准测试程序进行应用
\item 项目被评为首都师范大学``实验室开放基金"优秀项目(学院\textbf{唯一}一个被评为优秀的项目)
\end{itemize}
}
%------------------------------------------------
\subsection{``国家大学生创新性实验计划"项目}

\cventry{2010--2011}{项目成员}{Unix/Linux环境下路由管理转换接口设计与实现}{国家级}{}{
\begin{itemize}
\item 负责在Linux环境下实现软路由功能(基于GNU Zebra),设计实现路由管理转换接口
\end{itemize}
}

%------------------------------------------------
%%------------------------------------------------
%\subsection{首都师范大学学位论文\LaTeX 模板开发}
%\cventry{2013}{项目负责人}{首都师范大学学位论文(本科生、硕博)\LaTeX 模板}{开源项目}{}{
%\begin{itemize}
%	\item 开发和维护首都师范大学本科生,硕士生、博士生学位论文\LaTeX 模板
%\end{itemize}
%}

%----------------------------------------------------------------------------------------
%	AWARDS SECTION
%----------------------------------------------------------------------------------------

\section{荣誉奖励}

\cvitem{2013}{首都师范大学优秀毕业生}
\cvitem{2012}{2012年12月作为大学生创新性实验优秀学生赴韩国校外访学一周}
\cvitem{2011}{北京市大学生计算机应用大赛移动终端应用创意与程序设计二等奖~团队组长}
\cvitem{2011}{第六届全国信息技术应用水平大赛比赛安卓应用开发团体赛三等奖~团队组长}
\cvitem{2010 - 2012}{2010、2011、2012 学年度国家励志奖学金}
\end{document}
% 最后更新:2013年8月17日 10:18:45
